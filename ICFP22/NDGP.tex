\documentclass[acmsmall,review]{acmart}

%% Rights management information.  This information is sent to you
%% when you complete the rights form.  These commands have SAMPLE
%% values in them; it is your responsibility as an author to replace
%% the commands and values with those provided to you when you
%% complete the rights form.
\setcopyright{none}
%\setcopyright{acmcopyright}
%\copyrightyear{2018}
%\acmYear{2018}
%\acmDOI{10.1145/1122445.1122456}

%%
%% These commands are for a JOURNAL article.
\acmJournal{PACMPL}
\acmVolume{0}
\acmNumber{0}
\acmArticle{0}
\acmMonth{0}

%%
%% Submission ID.
%% Use this when submitting an article to a sponsored event. You'll
%% receive a unique submission ID from the organizers
%% of the event, and this ID should be used as the parameter to this command.
%%\acmSubmissionID{123-A56-BU3}

%%
%% The majority of ACM publications use numbered citations and
%% references.  The command \citestyle{authoryear} switches to the
%% "author year" style.
%%
%% If you are preparing content for an event
%% sponsored by ACM SIGGRAPH, you must use the "author year" style of
%% citations and references.
%% Uncommenting
%% the next command will enable that style.
\citestyle{acmauthoryear}

\settopmatter{printacmref=false}

\usepackage[color=yellow,textsize=footnotesize]{todonotes}

%%
%% end of the preamble, start of the body of the document source.
\begin{document}

%%
%% The "title" command has an optional parameter,
%% allowing the author to define a "short title" to be used in page headers.
\title{Naturalising Datatype-Generic Programs with Reflection}

%%
%% The "author" command and its associated commands are used to define
%% the authors and their affiliations.
%% Of note is the shared affiliation of the first two authors, and the
%% "authornote" and "authornotemark" commands
%% used to denote shared contribution to the research.
\author{Hsiang-Shang Ko}
\email{joshko@iis.sinica.edu.tw}
\orcid{0000-0002-2439-1048}
\author{Tzu-Chi Lin}
\email{vik@iis.sinica.edu.tw}
\author{Liang-Ting Chen}
\email{ltchen@iis.sinica.edu.tw}
\orcid{0000-0002-3250-1331}
\affiliation{%
  \institution{Institute of Information Science, Academia Sinica}
  \streetaddress{128 Academia Road, Section 2, Nankang}
  \city{Taipei}
  \country{Taiwan}
  \postcode{115201}
}

%%
%% By default, the full list of authors will be used in the page
%% headers. Often, this list is too long, and will overlap
%% other information printed in the page headers. This command allows
%% the author to define a more concise list
%% of authors' names for this purpose.
%\renewcommand{\shortauthors}{Trovato and Tobin, et al.}

%%
%% The abstract is a short summary of the work to be presented in the
%% article.
\begin{abstract}
\end{abstract}

%%
%% The code below is generated by the tool at http://dl.acm.org/ccs.cfm.
%% Please copy and paste the code instead of the example below.
%%
%\begin{CCSXML}
%<ccs2012>
% <concept>
%  <concept_id>10010520.10010553.10010562</concept_id>
%  <concept_desc>Computer systems organization~Embedded systems</concept_desc>
%  <concept_significance>500</concept_significance>
% </concept>
% <concept>
%  <concept_id>10010520.10010575.10010755</concept_id>
%  <concept_desc>Computer systems organization~Redundancy</concept_desc>
%  <concept_significance>300</concept_significance>
% </concept>
% <concept>
%  <concept_id>10010520.10010553.10010554</concept_id>
%  <concept_desc>Computer systems organization~Robotics</concept_desc>
%  <concept_significance>100</concept_significance>
% </concept>
% <concept>
%  <concept_id>10003033.10003083.10003095</concept_id>
%  <concept_desc>Networks~Network reliability</concept_desc>
%  <concept_significance>100</concept_significance>
% </concept>
%</ccs2012>
%\end{CCSXML}
%
%\ccsdesc[500]{Computer systems organization~Embedded systems}
%\ccsdesc[300]{Computer systems organization~Redundancy}
%\ccsdesc{Computer systems organization~Robotics}
%\ccsdesc[100]{Networks~Network reliability}

%%
%% Keywords. The author(s) should pick words that accurately describe
%% the work being presented. Separate the keywords with commas.
%\keywords{datasets, neural networks, gaze detection, text tagging}

%%
%% This command processes the author and affiliation and title
%% information and builds the first part of the formatted document.
\maketitle

\section{Introduction}

\citet{Gibbons-DGP, Altenkirch-GP-within-DTP, Pickering-staged-SoP, Chapman-type-theory-should-eat-itself}

\section{A First Taste of Datatype-Generic Programming}

\todo[inline]{Sums-of-products datatypes and inductive functions (as predicate algebras); `exotic' fixed points (and their problems); parametrisation}

\section{Datatype-Generic Programming with Inductive Families}

\section{Native Fixed Points}

\todo[inline]{Introduction to reflection in Agda; manufacturing native datatypes and inductive functions from datatype-generic definitions}

\section{Universe Polymorphism}

\section{Examples}

\todo[inline]{Derived datatypes and functions; ornamentation; specialised universes}

\section{Discussion}

%%
%% The acknowledgments section is defined using the "acks" environment
%% (and NOT an unnumbered section). This ensures the proper
%% identification of the section in the article metadata, and the
%% consistent spelling of the heading.
\begin{acks}
\end{acks}

%%
%% The next two lines define the bibliography style to be used, and
%% the bibliography file.
\bibliographystyle{ACM-Reference-Format}
\bibliography{/Users/joshko/Documents/bib}

%%
%% If your work has an appendix, this is the place to put it.
%\appendix
%
%\section{Research Methods}

\end{document}
